\documentclass[UTF8]{ctexart}
 
\CTEXsetup[format={\Large\bfseries}]{section}
\usepackage{amsmath}
\usepackage{algorithm}
\usepackage{algorithmic}

\renewcommand{\algorithmicrequire}{ \textbf{Input:}} %Use Input in the format of Algorithm
\renewcommand{\algorithmicensure}{ \textbf{Output:}} %UseOutput in the format of Algorithm
% 将require 和 ensure 改成了input 和 output.
 
\title{提升方法AdaBoost法1:算法介绍}
 
\begin{document}
 
\maketitle
 
\section{基本思路与概念}

\subsection{强可学习} 在概率近似正确(probably approximately correct,PAC)学习框架中,如果一个类存在一个多项式算法能够学习它,并且正确率很好,那么就称这个概念为强可学习。
\subsection{弱可学习} 相反,如果能够学习它,但是正确率仅仅比随机猜测好一点点,那么就称这个概念为弱可学习。
\subsection{弱和强是等价的} Schapire 证明了强可学习和弱可学习是等价的。



\subsection{两个问题}
对于分类而言,给定一个训练样本,求比较粗糙的分类算法比求更加精确的分类算法要简单很多。提升算法就是从弱分类出发,然后构造一系列弱分类算法,并将这些弱分类器组合成一个强分类器。
这一系列弱分类器大多是通过改变样本的分布或者称为权值(一个意思),来构建。

\subsubsection{问题1}
每一轮如何改变权值或者分布?
\subsubsection{问题2}
最后如何将这些弱分类器进行组合,使其称为强分类器?


\section{AdaBoost算法}
假设为二分类问题。

\begin{algorithm}[htb] 
	    	\caption{ AdaBoost算法.} 
	    	\label{alg:Framwork} 
	    	\begin{algorithmic}[1] %这个1 表示每一行都显示数字
	    		\REQUIRE ~~\\ %算法的输入参数:Input
	%    		The set of positive samples for current batch, $P_n$;\\
	%    		The set of unlabelled samples for current batch, $U_n$;\\
	%    		Ensemble of classifiers on former batches, $E_{n-1}$;
	    		训练数据集$T=\{(x_1,y_1),(x_2,y_2),...,(x_N,y_N)\}$其中$x_i \in X=R^n,y_i \in Y=\{-1,+1\},i=1,2,...,N $;弱学习算法$G_1(x)$\\
	    		
	    		
	    		\ENSURE ~~\\ %算法的输出:Output
	%    		Ensemble of classifiers on the current batch, $E_n$;
				强学习算法$F(x)$。\\
	
	
	%    		\STATE Extracting the set of reliable negative and/or positive samples $T_n$ from $U_n$ with help of $P_n$; 
	%    		\label{ code:fram:extract }%对此行的标记,方便在文中引用算法的某个步骤
	%    		\STATE Training ensemble of classifiers $E$ on $T_n \cup P_n$, with help of data in former batches; 
	%    		\label{code:fram:trainbase}
	%    		\STATE $E_n=E_{n-1}\cup E$; 
	%    		\label{code:fram:add}
	%    		\STATE Classifying samples in $U_n-T_n$ by $E_n$; 
	%    		\label{code:fram:classify}
	%    		\STATE Deleting some weak classifiers in $E_n$ so as to keep the capacity of $E_n$; 
	%    		\label{code:fram:select}
				\STATE 初始化权值分布
				$$
				D_1 = (w_{1,1},w_{1,2},...,w_{1,N}), \quad w_{1,i} = \frac{1}{N},\quad i = 1,2,...,N
				$$ 
				其中$D_m$表示第m次训练弱分类器对应的样本分布(权值),$w_{m,i}$ 表示每个样本的权值,初始化为均匀初始化。
				
				
				\STATE 对每一次(m)的弱分类器训练都执行如下步骤:\\
				(A).\\
				使用当前数据分布$D_m$的数据,训练弱分类器$G_m(x)$
				$$
				G_m(x):X \rightarrow \{1,-1\}
				$$
				
				(B).\\
				计算所得弱分类器$G_m(x)$的分类误差:
				$$
				\begin{aligned}
				 e_m &= \sum_{i=1}^{N}P(G_m(x_i) \ne y_i)\\
				 &= \sum_{i=1}^{N}w_{wi} I(G_m(x_i) \ne y_i)\\
				 &= \sum_{G_m(x_i \neq y_i)}w_i\\
				\end{aligned}
				$$
				误差率就是将分类错误的那些样本的权重相加即可。
				
				(C).\\
				由误差率计算本次的弱分类器在最终分类器中的系数$\alpha_m$,其表示在最终该弱分类器$G_m(x)$在最终分类器中的重要程度,准确率越高权重越大。
				$$
				\alpha_m = \frac{1}{2} \log{\frac{1-e_m}{e_m}}
				$$
				可以看出,是由误差率求出的。
				\\
				得出本次的 $f_m(x) = \alpha_m G_m(x)$
				(D).\\
				为下次训练$G_{m+1}(x)$调整数据的权重或者分布,得到$D_{m+1}$,调整策略根据样本预测的误差率来进行,分错的权重调大,分正确的权重调小。
				$$
				w_{m+1,i} =\frac{w_{m,i}}{Z_m}  \exp {(-\alpha_m {y_i G_m(x_i)})},\quad i=1,2,...,N
				$$
				其中可以看出,${y_i G_m(x_i)}$在分正确的情况下是等于-1的,分错误等于1。最终会将分正确的样本权重下调,错误的权重上调。
				
				
				\STATE 构建基本分类器的线性组合:
				$$
				\begin{aligned}
				F(x)  &= \sum_{m=1}^{M} \alpha_m G_m(x)  \\
				&= \sum_{i=1}^{N} f_i(x)    \\
				\end{aligned}
				$$
				如果$F(x)$的分类错误率达到要求,则终止,否则继续第2部,直到$F(x)$ 达到要求。\\
				\STATE OVER
	    	\end{algorithmic}
  \end{algorithm}


\begin{algorithm}[htb]
	\begin{algorithmic}[l]
	\REQUIRE 
	\ENSURE
	\STATE 如果$F(x)$的分类错误率达到要求,则终止,否则继续第2部,直到$F(x)$ 达到要求。\\
\STATE OVER
	\end{algorithmic}
\end{algorithm}





%    \begin{algorithm}
%    	\caption{非线性支持向量机学习算法} 
%    	\label{alg1}
%    	\begin{algorithmic}
%    		
%    		\REQUIRE $n \geq 0 \vee x \neq 0$ 
%    		\ENSURE $y = x^n$ 
%    		\STATE $y \Leftarrow 1$ 
%    		\IF{$n < 0$} 
%    		\STATE $X \Leftarrow 1 / x$ 
%    		\STATE $N \Leftarrow -n$ 
%    		\ELSE 
%    		\STATE $X \Leftarrow x$ 
%    		\STATE $N \Leftarrow n$
%    		\ENDIF 
%    		\WHILE{$N \neq 0$} 
%    		\IF{$N$ is even} 
%    		\STATE $X \Leftarrow X \times X$ 
%    		\STATE $N \Leftarrow N / 2$ 
%    		\ELSE[$N$ is odd] 
%    		\STATE $y \Leftarrow y \times X$ 
%    		\STATE $N \Leftarrow N - 1$ 
%    		\ENDIF 
%    		\ENDWHILE
%    	\end{algorithmic}
%    \end{algorithm}
% 
%\section{标题二}
%    \begin{algorithm}[htb] 
%    	\caption{ 非线性支持向量机学习算法.} 
%    	\label{alg:Framwork} 
%    	\begin{algorithmic}[1] %这个1 表示每一行都显示数字
%    		\REQUIRE ~~\\ %算法的输入参数:Input
%%    		The set of positive samples for current batch, $P_n$;\\
%%    		The set of unlabelled samples for current batch, $U_n$;\\
%%    		Ensemble of classifiers on former batches, $E_{n-1}$;
%    		训练数据集$T=\{(x_1,y_1),(x_2,y_2),...,(x_N,y_N)\}$其中$x_i \in X=R^n,y_i \in Y=\{-1,+1\},i=1,2,...,N $;\\
%    		
%    		
%    		\ENSURE ~~\\ %算法的输出:Output
%%    		Ensemble of classifiers on the current batch, $E_n$;
%			分类决策函数。\\
%
%
%%    		\STATE Extracting the set of reliable negative and/or positive samples $T_n$ from $U_n$ with help of $P_n$; 
%%    		\label{ code:fram:extract }%对此行的标记,方便在文中引用算法的某个步骤
%%    		\STATE Training ensemble of classifiers $E$ on $T_n \cup P_n$, with help of data in former batches; 
%%    		\label{code:fram:trainbase}
%%    		\STATE $E_n=E_{n-1}\cup E$; 
%%    		\label{code:fram:add}
%%    		\STATE Classifying samples in $U_n-T_n$ by $E_n$; 
%%    		\label{code:fram:classify}
%%    		\STATE Deleting some weak classifiers in $E_n$ so as to keep the capacity of $E_n$; 
%%    		\label{code:fram:select}
%			\STATE 选取适当的核函数$K(x,z)$和适当的参数$C$,构造并求解最优化问题
%			$$
%			\min_{\alpha} \quad \frac{1}{2}\sum_{i=1}^{N}\sum_{j=1}^{N}a_ia_jy_iy_jK(x_i,x_j)-\sum_{i=1}^{N}$$
%			$$s.t. \quad \sum_{i=1}^{N}\alpha_iy_i=0$$
%			$$\quad \quad 0 \le \alpha_i \le C, \quad i=1,2,...,N
%			$$
%			
%			求得最优解 $\alpha^* = (\alpha^*_1,\alpha^*_2,...,\alpha^*_N)^T$
%			\STATE 选取$\alpha^*$的一个正分量 $0 \textless \alpha^*_j \textless C$,计算
%			$$
%			b^* = y_i - \sum_{i=1}^{N}\alpha^*y_iK(x_i,x_j)
%			$$
%			\STATE 构造决策函数:
%			$$
%			f(x) = sign(\sum_{i=1}^{N}\alpha^*_i y_i K(x,x_i) + b^*)
%			$$
%			当$K(x,z)$是正定核函数,问题为凸规划问题,解是存在的,并且是最优解。
%%    		\RETURN $E_n$; %算法的返回值
%    	\end{algorithmic}
%    \end{algorithm}
%
%
%
%
%\section{标题二}
%\begin{algorithm}[htb] 
%	\caption{SMO算法.} 
%	\label{alg:Framwork} 
%	\begin{algorithmic}[1] %这个1 表示每一行都显示数字
%		\REQUIRE ~~\\ %算法的输入参数:Input
%		%    		The set of positive samples for current batch, $P_n$;\\
%		%    		The set of unlabelled samples for current batch, $U_n$;\\
%		%    		Ensemble of classifiers on former batches, $E_{n-1}$;
%		训练数据集$T=\{(x_1,y_1),(x_2,y_2),...,(x_N,y_N)\}$其中$x_i \in X=R^n,y_i \in Y=\{-1,+1\},i=1,2,...,N ,$精度$\epsilon$;\\
%		
%		
%		\ENSURE ~~\\ %算法的输出:Output
%		%    		Ensemble of classifiers on the current batch, $E_n$;
%		近似解$\hat{\alpha}$\\
%		
%		
%		%    		\STATE Extracting the set of reliable negative and/or positive samples $T_n$ from $U_n$ with help of $P_n$; 
%		%    		\label{ code:fram:extract }%对此行的标记,方便在文中引用算法的某个步骤
%		%    		\STATE Training ensemble of classifiers $E$ on $T_n \cup P_n$, with help of data in former batches; 
%		%    		\label{code:fram:trainbase}
%		%    		\STATE $E_n=E_{n-1}\cup E$; 
%		%    		\label{code:fram:add}
%		%    		\STATE Classifying samples in $U_n-T_n$ by $E_n$; 
%		%    		\label{code:fram:classify}
%		%    		\STATE Deleting some weak classifiers in $E_n$ so as to keep the capacity of $E_n$; 
%		%    		\label{code:fram:select}
%		\STATE 初值$\alpha^(0)=0$,令$k=0$,表示第0次更新的$\alpha$值;
%		\STATE 选取优化变量$\alpha_1^k,\alpha_2^k,$求解两个变量的最优化问题的解析解,得到$\alpha_1^{(k+1)},\alpha_2^{(k+2)}$,用他们更新$\alpha$为$\alpha^{(k+1)}$.
%		
%	
%		\STATE 若在精度$\epsilon$范围内满足停机条件
%		$$
%		\sum_{i=1}^{N}\alpha_i y_i = 0, \quad 0 \le \alpha_i \le C, \quad i = 1,2,...,N
%		$$
%		
%		
%		$$	
%y_i \cdot g(x_i)\left\{\begin{array}{l}
%	\geq 1,\left\{ x_i \mid \alpha_i = 0 \right\} \\
% = 1,\left\{x_i|0\textless \alpha_i \textless C   \right\}\\
%	\le 1,\left\{   x_i|\alpha_i = C  \right\}
%		\end{array}\right.
%		$$
%	其中,
%	$$
%	g(x_i) = \sum_{j=1}^{N}\alpha_j y_j K(x_j,x_i) + b
%	$$
%	
%	$$
%	b_1^{new} = -E_1 - y_1 K_{11}(\alpha_1^{new} - \alpha_1^{old} - y_2 K_{21}(\alpha_2^{new}- \alpha_2^{old}) + b^{old})
%	$$
%	$$
%	b_2^{new} = -E_2 - y_1 K_{12}(\alpha_1^{new} - \alpha_1^{old} - y_2 K_{22}(\alpha_2^{new}- \alpha_2^{old}) + b^{old})
%	$$
%	如果 $\alpha_1^{new},\alpha_2^{new}$同时满足$0 < \alpha_i^{new} < C,i=1,2$,那么$b_1^{new}=b_2^{new}$,否则取$b^{new} = \frac{b_1^{new}+b_2^{new}}{2} $\\
%	然后更新$E_i$:
%	$$
%	E_i^{new} = \sum_{S}y_i \alpha_i K{(x_i,x_j)} + b^{new} - y_i
%	$$
%	
%	
%	则转到(4);否则令$k = k + 1$,转到(2);
%		\STATE 取$\hat{\alpha}=\alpha^{(k+1)}$
%	
%	\end{algorithmic}
%\end{algorithm}
\end{document}
