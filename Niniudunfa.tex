\documentclass[UTF8]{ctexart}
 
\CTEXsetup[format={\Large\bfseries}]{section}
\usepackage{amssymb}
\usepackage{amsmath}
\usepackage{algorithm}
\usepackage{algorithmic}

\renewcommand{\algorithmicrequire}{ \textbf{Input:}} %Use Input in the format of Algorithm
\renewcommand{\algorithmicensure}{ \textbf{Output:}} %UseOutput in the format of Algorithm
% 将require 和 ensure 改成了input 和 output.
 \newtheorem{thm}{\bf Theorem}[section]
 % 定理。
 
 
\title{机器学习笔记(5):牛顿法和拟牛顿法}
\author{https://blog.csdn.net/gaocui883}
\begin{document}
 
\maketitle
 
\section{牛顿法}
牛顿法和拟牛顿法是求解无约束最优化问题的常用方法,收敛速度快。牛顿法需要求解黑塞矩阵的逆矩阵,计算比较复杂。

拟牛顿法通过正定矩阵近似黑塞矩阵的逆矩阵或者黑塞矩阵本身,简化了计算过程。




无约束优化:$\min _{x \in \mathbf{R}^{n}} f(x)$

二阶泰勒展开:

$f(x)=f\left(x^{(k)}\right)+g_{k}^{\mathrm{T}}\left(x-x^{(k)}\right)+\frac{1}{2}\left(x-x^{(k)}\right)^{\mathrm{T}} H\left(x^{(k)}\right)\left(x-x^{(k)}\right)$\\


其中: $g_{k}=g\left(x^{(k)}\right)=\nabla f\left(x^{(k)}\right)$

Hesse matrix : 
$$H(x)=\left[\frac{\partial^{2} f}{\partial x_{i} \partial x_{j}}\right]_{n \times n}$$

极小值的必要条件:
$
\begin{array}{c}
	\nabla f(x)=0
\end{array}
$

每次从$x^k$迭代,假设$x^{k+1}$ 满足:
$$
\nabla f\left(x^{(k+1)}\right)=0
$$
利用泰勒展开的结果:
$$
\nabla f(x)=g_{k}+H_{k}\left(x-x^{(k)}\right) = 0
$$
$$
\begin{array}{c}
x^{(k+1)}=x^{(k)}-H_{k}^{-1} g_{k} \\
x^{(k+1)}=x^{(k)}+p_{k} \\
H_{k} p_{k}=-g_{k}
\end{array}
$$
用$x^k$与$x^{k+1}$不停的迭代,知道$g_{k}$逐渐真正的接近于0.

\begin{algorithm}[htb] 
	    	\caption{ 牛顿法:} 
	    	\label{alg:Framwork} 
	    	\begin{algorithmic}[1] %这个1 表示每一行都显示数字
	    		\REQUIRE ~~\\ %算法的输入参数:Input
	%    		The set of positive samples for current batch, $P_n$;\\
	%    		The set of unlabelled samples for current batch, $U_n$;\\
	%    		Ensemble of classifiers on former batches, $E_{n-1}$;
	    		目标函数$f(x)$,梯度$g(x)=\nabla f(x)$, Hesse矩阵$H(x)$, 精度要求$\epsilon$
	    		
	    		\ENSURE ~~\\ %算法的输出:Output
	%    		Ensemble of classifiers on the current batch, $E_n$;
				$f(x)$的极小值点$x^*$。\\
				\STATE 取初始值 $x^0$, $k=0$
				\STATE 计算$g_{k} = \nabla f(x^{k})$
				\STATE 如果$||g_{k}|| < \epsilon $,则停止计算,当前的 $x^{k}$ 就是 $x^{*}$
				\STATE 计算黑塞矩阵$H(x)$, 求$ p_{k}$
				$$
				H_{k} p_{k}=-g_{k}
				$$
				\STATE $x^{k+1} = x^{k} +  p_{k}$
				\STATE $k = k+1$ 重复到第二部,求$g_{k} = \nabla f(x^{k})$
				
	
	    	\end{algorithmic}
  \end{algorithm}

牛顿法计算复杂之处在于求黑塞矩阵和黑塞矩阵的逆矩阵,而拟牛顿法的思路就是在这上边进行改进。

\section{拟牛顿法}
拟牛顿法的思路是用一个n阶矩阵 $G_k = G(x^k)$ 来近似相应的黑塞矩阵的逆矩阵$H^{-1}(x^{k})$;或者用一个n阶矩阵$B_k = B(x^k)$来近似黑塞矩阵本身$H(x^{k})$

$$
\nabla f(x)=g_{k}+H_{k}\left(x-x^{(k)}\right) = 0
$$
得到:
$$
g_{(k+1)} = \nabla f(x+1)=g_{k}+H_{k}\left(x^{k+1}-x^{(k)}\right) = 0
$$
$$
g_{(k+1)}-g_{k} = H_{k}\left(x^{k+1}-x^{(k)}\right)
$$
设:$y_k = g_{(k+1)}-g_{k}, \delta_k = \left(x^{k+1}-x^{(k)}\right)$\\
有:
$$
y_k = H_{k}\delta_k
$$
$$H_{k}^{-1} y_{k}=\delta_{k}$$

当$H_{k}$ 为正定的,那么可以保证$f(x)$是下降方向的,也就是会不停的减少的。

寻找一个$G_k$ 满足 : 
$G_{k+1} y_{k}=\delta_{k}$ 并且正定即可。

\subsection{DFP算法}
构造:$G_{k+1}=G_{k}+P_{k}+Q_{k}$\\
$$G_{k+1} y_{k}=G_{k} y_{k}+P_{k} y_{k}+Q_{k} y_{k}$$
让其中的
$$\begin{array}{c}
P_{k} y_{k}=\delta_{k} \\
Q_{k} y_{k}=-G_{k} y_{k}
\end{array}$$ 即可得到最终的$G_{k+1} y_{k}=\delta_{k}$

构造Pk, Qk:
$$\begin{array}{c}
P_{k}=\frac{\delta_{k} \delta_{k}^{\mathrm{T}}}{\delta_{k}^{\mathrm{T}} y_{k}} \\
Q_{k}=-\frac{G_{k} y_{k} y_{k}^{\mathrm{T}} G_{k}}{y_{k}^{\mathrm{T}} G_{k} y_{k}}
\end{array}$$
可以满足上述的要求。

最终得到了迭代公式:$G_{k+1}=G_{k}+\frac{\delta_{k} \delta_{k}^{\mathrm{T}}}{\delta_{k}^{\mathrm{T}} y_{k}}-\frac{G_{k} y_{k} y_{k}^{\mathrm{T}} G_{k}}{y_{k}^{\mathrm{T}} G_{k} y_{k}}$

\begin{algorithm}[h] 
	\caption{ 拟牛顿法DFP算法:} 
	\label{alg:Framwork2} 
	\begin{algorithmic}[1] %这个1 表示每一行都显示数字
		\REQUIRE ~~\\ %算法的输入参数:Input
		%    		The set of positive samples for current batch, $P_n$;\\
		%    		The set of unlabelled samples for current batch, $U_n$;\\
		%    		Ensemble of classifiers on former batches, $E_{n-1}$;
		目标函数$f(x)$,梯度$g(x)=\nabla f(x)$, Hesse矩阵$H(x)$, 精度要求$\epsilon$
		
		\ENSURE ~~\\ %算法的输出:Output
		%    		Ensemble of classifiers on the current batch, $E_n$;
		$f(x)$的极小值点$x^*$。\\
		\STATE 取初始值 $x^0$, $k=0$,取正定对称矩阵$G_0 = I$.
		\STATE 计算$g_{k} = \nabla f(x^{k})$, 如果$||g_{k}|| < \epsilon $,则停止计算,当前的 $x^{k}$ 就是 $x^{*}$.
		\STATE 计算$p_k = -G_k g_k$,其中的$p_k$ 的计算用GK代替了牛顿法中的黑塞矩阵。
		$$
		x=x^{(k)}+\lambda p_{k}=x^{(k)}-\lambda H_{k}^{-1} g_{k}
		$$ $$
		f(x)=f\left(x^{(k)}\right)-\lambda g_{k}^{\mathrm{T}} H_{k}^{-1} g_{k}
		$$
		第二个公式是从开始的泰勒展开推出来的,表示正定矩阵条件下,f(x)必定会越来越小。
		\STATE 一维搜索:求$\lambda_k$ 
		$$
		f\left(x^{(k)}+\lambda_{k} p_{k}\right)=\min _{\lambda \geqslant 0} f\left(x^{(k)}+\lambda p_{k}\right)
		$$
		
		此时的$x^{(k+1)} = x^{(k)} + \lambda_k p_{k}$
		\STATE 计算此时的$g_{k+1}$,如果达到要求,则停止,当前的 $x^{k+1}$ 就是 $x^{*}$,否则求
		$G_{k+1}=G_{k}+\frac{\delta_{k} \delta_{k}^{\mathrm{T}}}{\delta_{k}^{\mathrm{T}} y_{k}}-\frac{G_{k} y_{k} y_{k}^{\mathrm{T}} G_{k}}{y_{k}^{\mathrm{T}} G_{k} y_{k}}$
		,设置k=k+1.继续到$p_k = -G_k g_k$,继续迭代,直到收敛。
		
	
		
		
	\end{algorithmic}
\end{algorithm}


\subsection{DFGS算法}
DFP算法使用直接逼近$H^-1_k$矩阵的,如果改成逼近$H_k$矩阵的$B_k$矩阵,那么就是DFGS算法。
$$B_{k+1} \delta_{k}=y_{k}$$
$$B_{k+1}=B_{k}+P_{k}+Q_{k}$$
$$B_{k+1} \delta_{k}=B_{k} \delta_{k}+P_{k} \delta_{k}+Q_{k} \delta_{k}$$
$$\begin{array}{c}
P_{k} \delta_{k}=y_{k} \\
Q_{k} \delta_{k}=-B_{k} \delta_{k}\\
B_{k+1}=B_{k}+\frac{y_{k} y_{k}^{\mathrm{T}}}{y_{k}^{\mathrm{T}} \delta_{k}}-\frac{B_{k} \delta_{k} \delta_{k}^{\mathrm{T}} B_{k}}{\delta_{k}^{\mathrm{T}} B_{k} \delta_{k}}
\end{array}$$
现在问题是,虽然$B_k$可以逼近$H_k$但是其实求解迭代时候还是需要用$H^{-1}_k$ 或者$G_{k}$.

我们可以得知,$B_k$ 可以可以迭代得到,对于这种矩阵的逆矩阵可以使用Sherman-Morrision公式:
$$\left(A+u v^{\mathrm{T}}\right)^{-1}=A^{-1}-\frac{A^{-1} u v^{\mathrm{T}} A^{-1}}{1+v^{\mathrm{T}} A^{-1} u}$$
得到G的迭代公式:
$$G_{k+1}=\left(I-\frac{\delta_{k} y_{k}^{\mathrm{T}}}{\delta_{k}^{\mathrm{T}} y_{k}}\right) G_{k}\left(I-\frac{\delta_{k} y_{k}^{\mathrm{T}}}{\delta_{k}^{\mathrm{T}} y_{k}}\right)^{\mathrm{T}}+\frac{\delta_{k} \delta_{k}^{\mathrm{T}}}{\delta_{k}^{\mathrm{T}} y_{k}}$$
虽然形式与DFP方法所得到的形式不同,但是它是满足拟牛顿条件的。


\begin{algorithm}[h] 
	\caption{ 拟牛顿法DFGS算法:} 
	\label{alg:Framwork3} 
	\begin{algorithmic}[1] %这个1 表示每一行都显示数字
		\REQUIRE ~~\\ %算法的输入参数:Input
		%    		The set of positive samples for current batch, $P_n$;\\
		%    		The set of unlabelled samples for current batch, $U_n$;\\
		%    		Ensemble of classifiers on former batches, $E_{n-1}$;
		目标函数$f(x)$,梯度$g(x)=\nabla f(x)$, Hesse矩阵$H(x)$, 精度要求$\epsilon$
		
		\ENSURE ~~\\ %算法的输出:Output
		%    		Ensemble of classifiers on the current batch, $E_n$;
		$f(x)$的极小值点$x^*$。\\
		\STATE 取初始值 $x^0$, $k=0$,取正定对称矩阵$B_0 = I$.
		\STATE 计算$g_{k} = \nabla f(x^{k})$, 如果$||g_{k}|| < \epsilon $,则停止计算,当前的 $x^{k}$ 就是 $x^{*}$.
		\STATE 计算$B_k p_k =  g_k$,其中的$p_k$ 的计算用bK代替了牛顿法中的黑塞矩阵。
		
	
		\STATE 一维搜索:求$\lambda_k$ 
		$$
		f\left(x^{(k)}+\lambda_{k} p_{k}\right)=\min _{\lambda \geqslant 0} f\left(x^{(k)}+\lambda p_{k}\right)
		$$
		
		此时的$x^{(k+1)} = x^{(k)} + \lambda_k p_{k}$
		\STATE 计算此时的$g_{k+1}$,如果达到要求,则停止,当前的 $x^{k+1}$ 就是 $x^{*}$,否则求
		$B_{k+1}=B_{k}+\frac{y_{k} y_{k}^{\mathrm{T}}}{y_{k}^{\mathrm{T}} \delta_{k}}-\frac{B_{k} \delta_{k} \delta_{k}^{\mathrm{T}} B_{k}}{\delta_{k}^{\mathrm{T}} B_{k} \delta_{k}}
		$
		,设置k=k+1.继续到$p_k = -G_k g_k$,继续迭代,直到收敛。

	\end{algorithmic}
\end{algorithm}
\end{document}
